%!TEX encoding = UTF-8 Unicode
% Author: Laurent Dutriaux
\documentclass[a4paper,11pt]{article}
\usepackage[utf8]{inputenc}
\usepackage{fourier} % Utilisation des polices texte
\usepackage{tikz}
%%%<
\usepackage{verbatim}
\usepackage[active,tightpage]{preview}
\PreviewEnvironment{tikzpicture}
\setlength\PreviewBorder{5pt}%
%%%>

\usetikzlibrary[positioning]
\usetikzlibrary{patterns}
\usepackage[french]{babel} % styles français

\begin{document}
\begin{tikzpicture}
%\draw[help lines] (0,0) grid (18,11);
\filldraw[fill=blue!20] (0,7) rectangle +(1,4);
\filldraw[fill=blue!20] (1,4) rectangle +(1,3);
\filldraw[fill=blue!20] (2,0) rectangle +(1,4);
\draw[very thick] (0,0) rectangle +(3,11);

\filldraw[fill=blue!20,very thick] (3.5,8) rectangle +(3,3);
\draw (4.5,8) rectangle +(1,1);
\draw (3.5,9) rectangle +(1,1);
\draw (5.5,9) rectangle +(1,1);
\draw (4.5,10) rectangle +(1,1);


\filldraw[fill=blue!20] (7,10) rectangle +(4,1);
\filldraw[fill=blue!20] (11,9) rectangle +(3,1);
\filldraw[fill=blue!20] (14,8) rectangle +(4,1);


\draw[very thick] (7,8) rectangle +(11,3);


\end{tikzpicture}

\end{document}