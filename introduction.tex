\chapter{Introduction\index{Introduction}}


\textsf{%
In this chapter we introduce the field of Datamining and its relevance. We
also introduce the area of research in which this thesis work is done.}

\section{Introduction\index{Introduction}}
The field of Data Mining aka Knowledge Discovery in Data, or KDD saw its first
light during the 1990s, and has evolved along with the processing power and
storage capacities of modern computer systems. According to IBM, data is being
ubiquitously generated at a rate of 2.5 billion gigabytes each day
\cite{IBM_BigData:online}. The amount of data is exploding with as estimate of
35 zetabytes by the year 2020, which seems realistic with just social ineraction
websites like twitter contributing around 7 terabytes(TB) each day and facebook
generating around 10 terabytes \cite{IBM-BigData}. The analytics possibilities
with this vast amounts of data is again as diverse as the sources of data, in
the political arena, Big data technology played an important factor for Barack
Obama's 2012 presidential campaign, where information from previous election
were used \cite{BarackObama}. Researchers from Uppsala Monitoring Centre have
been able to create a comprehensive database of Adverse Drug Reactions(ADRs)
using advanced data mining techniques \cite{UppsalaMonitoring}. In the business
front, enables new business intelligence dimension, NETFLIX is using its user
ratings information to provide better predictions and thereby helping its
customers make choices. 

We intend to obtain/infer useful knowledge from raw data, by observing
underlying patterns which characterizes the system represented by the raw data.
In this complex process of infering knowledge from raw data, data mining forms
the part which requires a mathematical approach. The data mining stage is
preceded by certain engineering stages like creation and preparation of data. It
is succeeded by certain business or domain specific knowledge inference and
analysis stages. There are six main tasks within the gamut of data mining,
enlisted below, \cite{Fayyad96fromdata}
\begin{description}
  \item[Classification] is learning a function that maps a data item into one of
several predefined classes.
  \item[Regression] is learning a function that maps a data item to a
real-valued prediction variable.
  \item[Clustering] is a common descriptive task where one seeks to identify a
finite set of categories or clusters to describe the data.
  \item[Summarization] involves methods for finding a compact description for a
subset of data.
  \item[Dependency Modeling] consists of finding a model that describes
significant dependencies between variables.
\item[Change and deviation detection] focuses on discovering the most
significant changes in
the data from previously measured or normative values.
\end{description}

From a mathematical perspective data mining or pattern recognition can be viewed
as a data model fitting problem. The processed data what we use to build a
model upon is called as the \emph{training data}. This training data is prepared
from the factual previous knowledge. Next after building a model, we would like
to ensure its validity. There needs to be some kind of measurement to esnure
certain desirable degree of correctness of the model. This forms the \emph{error
analysis}, and metrics like \emph{RMSE} are used often. Linear algebra proves to
be pivotal in data mining, \cite{eld-mm:07} to build these models. 

In this master thesis work, our goal is to build a recommender system, which
will help people choose movies they would like to watch. For building the same,
we need some previous knowledge, an idea or some methodology on how to model
this data, and some measure for degree of accuracy of the model.

The kind of prediction we are dealing with is ratings prediction, which gives
an indication of user's preference to items. Depending on the predicted rating
recommendations are made, this involves different tasks the most significant of
them being matrix factorization and rank reduction. It is very common while
dealing with preferences that they tend to change with time, the work by
\cite{Koren:2010:CFT:1721654.1721677} models the data drifting with time. It is
important to note that while modeling temporal effects we should take into
account there are transient signals analogous to users and a more long term
patterns characterized by movies.


