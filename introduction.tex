\chapter{Introduction\index{Introduction}}


\textsf{%
In this chapter we introduce the field of Datamining and its relevance. We
also introduce the area of research in which this thesis work is done.}

\section{Introduction\index{Introduction}}
The field of Data Mining aka Knowledge Discovery in Data, or KDD saw its first
light during the 1990s, and has evolved along with the processing power and
storage capacities of modern computer systems. The amount of data only from the
web is in the order of exabytes and continues to increase, this along with other
sources of data resulting from reserach and experimentation proves to be one the
most important sources of knowledge. The data mining techniques are applied
to wide variety of applications like in business, to improve customer
relationhip management, or to do market analysis with respect to a particular
product. The medical fraternity use them to do temporal analysis of patterns
concerning drug presciptions to diagnoses. With respect to science and
engineering, DNA sequence mining proves very crucial in understanding the
variations in human DNA. Data mining actually can be seen as one step towards
discovering knowledge from raw source, initially there are many more processes
like understanding of the application domain and collecting the data, then there
is data transformation and only then is the actual data mining techniques
applied. Even after the data is mined and clusters or patterns are obtained,
interpretation of these patterns and taking required actions forms the last
stage. There are six main tasks within the gamut of data mining, enlisted
below, \cite{Fayyad96fromdata}
\begin{description}
  \item[Classification] is learning a function that maps a data item into one of
several predefined classes.
  \item[Regression] is learning a function that maps a data item to a
real-valued prediction variable.
  \item[Clustering] is a common descriptive task where one seeks to identify a
finite set of categories or clusters to describe the data.
  \item[Summarization] involves methods for finding a compact description for a
subset of data.
  \item[Dependency Modeling] consists of finding a model that describes
significant dependencies between variables.
\item[Change and deviation detection] focuses on discovering the most
significant changes in
the data from previously measured or normative values.
\end{description}

  The kind of prediction we are dealing with is ratings prediction, which gives
an indication of user's preference to items. Depending on the predicted rating
recommendations are made, this involves different tasks the most significant of
them being matrix factorization and rank reduction. It is very common while
dealing with preferences that they tend to change with time, the work by
\cite{Koren:2010:CFT:1721654.1721677} models the data drifting with time. It is
important to note that while modeling temporal effects we should take into
account there are transient signals analogous to users and a more long term
patterns characterized by movies.


